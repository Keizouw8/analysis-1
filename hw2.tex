\documentclass[letterpaper]{article}
\usepackage{amssymb}
\usepackage{amsmath}
\usepackage{titlesec}
\usepackage{marginnote}
\usepackage{mathtools}
\usepackage{changepage}

\title{Homework 2}
\author{Keizou Wang}
\date{February 12, 2025}

\titleformat{\section}{\normalfont\normalsize}{\bfseries\thesection.}{1em}{}
\titleformat{\subsection}{\normalfont\normalsize}{(\textit{\roman{subsection}})}{1em}{}

\begin{document}

\maketitle

\section{}
\subsection{Show that given finitely many countable sets $A_1,A_2,\dotsb,A_n$, the set $A_1\times A_2\times\dotsb\times A_n$ is also countable.}
\begin{gather*}
|A_k|=|\mathbb{N}|\Rightarrow\exists f_k:\mathbb{N}\rightarrow A_k \text{ is bijective} \\
f(x_1,x_2,\dotsb,x_n)=(f_1(x_1),f_2(x_2),\dotsb,f_n(x_n)): x_1,x_2,\dotsb,x_n\in\mathbb{N} \\
f:\mathbb{N}\times\mathbb{N}\times\dotsb\times\mathbb{N}\rightarrow A_1\times A_2\times\dotsb\times A_n\text{ is bijective} \\
\line(1, 0){125} \\
h_2(x_1, x_2)=2^{n-1}(2m-1): x_1,x_2\in\mathbb{N} \\
h_2: \mathbb{N}\times\mathbb{N}\rightarrow\mathbb{N}\text{ is bijective} \\
h_3(x_1,x_2,x_3)=h_2(h_2(x_1,x_2),x_3):x_1,x_2,x_3\in\mathbb{N}
h_2(x_1,x_2) \\
h_3: \mathbb{N}\times\mathbb{N}\times\mathbb{N}\rightarrow\mathbb{N}\text{ is bijective} \\
h_k(x_1,x_2,\dotsb,x_k)=h_{k-1}(h_{k-1}(x_1,x_2,\dotsb,x_{k-1}),x_k):x_1,x_2,\dotsb,x_k\in\mathbb{N} \\
h_k:\mathbb{N}\times\mathbb{N}\times\dotsb\times\mathbb{N}\rightarrow\mathbb{N}\text{ is bijective} \\
\text{Let }g_n=h_n^{-1}\Rightarrow g_n:\mathbb{N}\rightarrow\mathbb{N}\times\mathbb{N}\times\dotsb\times\mathbb{N}\text{ is bijective} \\
\line(1, 0){125} \\
f\circ g: \mathbb{N}\rightarrow A_1\times A_2\times\dotsb\times A_n\text{ is bijective} \\
\therefore \\
\boxed{|A_1\times A_2\times\dotsb\times A_n|=|\mathbb{N}|}
\end{gather*}
\subsection{Is it true that given countably many countable sets $A_1,A_2,\dotsb$, the set $A_1\times A_2\times\dotsb$ is also countable? Justify your answer.}
\begin{center}
Assume bijection, all output tuples can be represented as:\marginnote{$f_{nm}$ as in section (\textit{i}) is bijective between some $A_k$ and $\mathbb{N}$} \newline\newline
\begin{tabular}{c c c c c}
$(f_{11}(x_{11})$ & $f_{12}(x_{12})$ & $\dotsb$ & $f_{1m}(x_{1m})$ & $\dotsb)$ \\
$(f_{21}(x_{21})$ & $f_{22}(x_{22})$ & $\dotsb$ & $f_{2m}(x_{2m})$ & $\dotsb)$ \\
$(f_{n1}(x_{n1})$ & $f_{n2}(x_{n2})$ & $\dotsb$ & $f_{nm}(x_{nm})$ & $\dotsb)$
\end{tabular} \newline\newline
Consider the tuple: $R=(f_{11}(x_{11}+1),f_{22}(x_{22}+1),\dotsb,f_{nn}(x_{nn}+1),\dotsb)$ \\
$R$ will differ from every row by at least one value along the diagonal because $f_{nn}$ is bijective, so $a\neq b\Rightarrow f_{nn}(a)\neq f_{nn}(b)$, and $x_{nn}\neq x_{nn}+1$. However, $R\in A_1\times A_2\times\dotsb$ which means it should be indexable in the bijection, a contradiction. \\
$\newline\therefore$ \\
$\newline\boxed{|A_1\times A_2\times\dotsb\times A_n|\neq|\mathbb{N}|}$
\end{center}

\section{}
\subsection{Let $\mathcal{F}$ be the collection of all functions $f:\{0,1\}\rightarrow\mathbb{N}$. Is $\mathcal{F}$ countable? Justify your answer.}
\begin{gather*}
f\in\mathcal{F}\Rightarrow f(x)=\begin{cases}
n & \text{if } x=0:n\in\mathbb{N} \\
m & \text{if } x=1:m\in\mathbb{N}
\end{cases} \\
g=(n,m)\mapsto (x\mapsto\begin{cases}
n & \text{if } x=0 \\
m & \text{if } x=1
\end{cases}):n,m\in\mathbb{N} \\
g:\mathbb{N}\times\mathbb{N}\rightarrow\mathcal{F}\text{ is bijective} \\
|\mathcal{F}|=|\mathbb{N}\times\mathbb{N}|=|\mathbb{N}| \\
\therefore \\
\boxed{|\mathcal{F}|=|\mathbb{N}|}
\end{gather*}
\subsection{Let $\mathcal{F}$ be the collection of all functions $f:\mathbb{N}\rightarrow\{0,1\}$. Is $\mathcal{F}$ countable? Justify your answer.}
\begin{gather*}
F\subset\mathcal{F},F\in \{f(x)=\text{round}(a^x): n\in\mathbb{N},a\in(0, 1)\} \\
g=a\mapsto(x\mapsto\text{round}(a^x)):n\in\mathbb{N},a\in(0, 1) \\
g:(0,1)\rightarrow F\text{ is bijective} \\
|\mathcal{F}|\geq|F|=|(0,1)|=|\mathbb{R}|>|\mathbb{N}| \marginnote{$|(0,1)|=|\mathbb{R}|$ shown in \textbf{3}(\textit{ii})} \\
\therefore \\
\boxed{|\mathcal{F}|\neq|\mathbb{N}|}
\end{gather*}
\subsection{Let $\mathcal{F}$ be the collection of all functions $f:\mathbb{N}\rightarrow\mathbb{N}$. Is $\mathcal{F}$ countable? What about $f:\mathbb{R}\rightarrow\mathbb{R}$? Justify your answers.}
\begin{gather*}
F\subset\mathcal{F},F=\{f(n)=\lceil rn\rceil:n\in\mathbb{N},r\in\mathbb{R}\} \\
g=r\mapsto(n\mapsto\lceil rn\rceil):n\in\mathbb{N},r\in\mathbb{R} \\
g:\mathbb{R}\rightarrow F\text{ is bijective}\Rightarrow|F|=|\mathbb{R}|
\end{gather*}
\begin{gather*}
|\mathcal{F}|\geq|F|=|\mathbb{R}|>|\mathbb{N}| \\
\therefore \\
\boxed{|\mathcal{F}|\neq|\mathbb{N}|:\mathcal{F}\text{ set of all }f:\mathbb{N}\rightarrow\mathbb{N}} \\
\line(1, 0){125} \\
F\subset\mathcal{F},F=\{f(x)=rx:x\in\mathbb{R},r\in\mathbb{R}\} \\
g=r\mapsto(x\mapsto rx):x,r\in\mathbb{R} \\
g:\mathbb{R}\rightarrow F\text{ is bijective}\Rightarrow|F|=|\mathbb{R}| \\
|\mathcal{F}|\geq|F|=|\mathbb{R}|>|\mathbb{N}| \\
\therefore \\
\boxed{|\mathcal{F}|\neq|\mathbb{N}|:\mathcal{F}\text{ set of all }f:\mathbb{R}\rightarrow\mathbb{R}} \\
\end{gather*}

\section{}
\subsection{Let $(0, 1)$ be the open interval from 0 to 1. Show that $|(0, 1)|=|\mathbb{R}^+|$ by finding an explicit bijection.}
\begin{gather*}
\boxed{f(x)=\frac{1}{x+1}}
\end{gather*}
\subsection{Show that $|(0, 1)|=|\mathbb{R}|$ by finding an explicit bijection.}
\begin{gather*}
\boxed{f(x)=\frac{1}{\pi}\arctan(x)+\frac{1}{2}}
\end{gather*}
\subsection{Show that $|\mathcal{P}(\mathbb{N})|=|\mathbb{R}^+|$.}
\begin{gather*}
f(S)=\sum_{k=1}^{|S|}2^{-S_k}: S\in\mathcal{P}(\mathbb{N}) \marginnote{$S_k$ is $k^\text{th}$ element of $S$} \\
f:\mathcal{P}(\mathbb{N})\rightarrow (0,1) \text{ is bijective} \\
\therefore \\
\boxed{|\mathcal{P}(\mathbb{N})|=|(0,1)|=|\mathbb{R}^+|} \marginnote{$|(0,1)|=|\mathbb{R}^+|$ shown in (\textit{i})}
\end{gather*}

\section{A real number $x$ is said to be \textit{algebraic} (over $\mathbb{Q}$) if it satisfies some polynomial equation $a_nx^n+a_{n-1}x^{n-1}+\dotsb+a_1x+a_0=0$, where each $a_i\in\mathbb{Q},a_n\neq0$. If $x$ is not algebraic, it is called \textit{transcendental}.}
\subsection{Show that the set of all polynomials over $\mathbb{Q}$ is countable.}
\begin{gather*}
P=\{a_0+a_1x+a_2x^2\dotsb+a_nx^n:a_0,a_1,a_2,\dotsb,a_n\in\mathbb{Q}\} \marginnote{set of all polynomials over $\mathbb{Q}$} \\
f=(c_0,c_1,c_2,\dotsb,c_n)\mapsto((x)\mapsto c_0+c_1x+c_2x^2\dotsb+c_nx^n) \\
f: \mathbb{Q}\times\mathbb{Q}\times\dotsb\times\mathbb{Q}\rightarrow P \text{ is bijective} \\
\line(1, 0){125} \\
|\mathbb{Q}|=|\mathbb{N}|\Rightarrow\exists q:\mathbb{N}\rightarrow\mathbb{Q}\text{ is bijective} \\
g(x_0,x_1,\dotsb,x_n)=(q(x_0),q(x_1),\dotsb,q(x_n)) \\
g:\mathbb{N}\times\mathbb{N}\times\dotsb\times\mathbb{N}\rightarrow\mathbb{Q}\times\mathbb{Q}\times\dotsb\times\mathbb{Q}\text{ is bijective} \\
\line(1, 0){125} \\
|\mathbb{N}\times\mathbb{N}\times\dotsb\times\mathbb{N}|=|\mathbb{N}| \marginnote{proven in work of \textbf{1}(\textit{i})} \Rightarrow \exists h:\mathbb{N}\rightarrow\mathbb{N}\times\mathbb{N}\times\dotsb\times\mathbb{N}\text{ is bijective} \\
f\circ g\circ h:\mathbb{N}\rightarrow P\text{ is bijective} \\
\therefore \\
\boxed{|P|=|\mathbb{N}|}
\end{gather*}
\subsection{Prove that there are countably many algebraic numbers.}
\begin{adjustwidth}{2.35em}{0pt}
Let $A$ denote the set of all algebraic numbers. There are a countable amount of polynomials and each contains an arbitrarily large amount of roots (fundamental theory of algebra). Therefore, because the amounts of roots are finite the cardinality can't be more than $|\mathbb{N}|\times|\mathbb{N}|$. However there is at least one root for each polynomial so there are at least $|\mathbb{N}|$. $|\mathbb{N}|\leq|A|\leq|\mathbb{N}|\times|\mathbb{N}|=|\mathbb{N}|\therefore|A|=|\mathbb{N}|$.
\end{adjustwidth}
\subsection{Prove that there are uncountably many transcendental numbers.}
\begin{gather*}
\text{Let T denote the set of transcendental numbers:} \\
S\subset T, S=\{r\pi:r\in\mathbb{R}\} \\
s(r)=r\pi:r\in\mathbb{R} \\
s:\mathbb{R}\rightarrow S \text{ is bijective} \\
|T|\geq|S|=|\mathbb{R}|>|\mathbb{N}| \\
\boxed{|T|\neq|\mathbb{N}|}
\end{gather*}

\end{document}