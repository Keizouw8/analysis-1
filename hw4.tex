\documentclass[letterpaper]{article}
\usepackage{amssymb}
\usepackage{amsmath}
\usepackage{titlesec}
\usepackage{marginnote}
\usepackage{mathtools}
\usepackage{changepage}

\title{Homework 4}
\author{Keizou Wang}
\date{February 26, 2025}

\titleformat{\section}{\normalfont\normalsize}{\bfseries\thesection.}{1em}{}
\titleformat{\subsection}{\normalfont\normalsize}{(\textit{\roman{subsection}})}{1em}{}

\DeclarePairedDelimiter{\ceil}{\lceil}{\rceil}
\DeclarePairedDelimiter\abs{\lvert}{\rvert}

\begin{document}

\maketitle

\section{Consider the following continued fraction: $1+\frac{1}{2+\frac{1}{2+\frac{1}{2+\dotsb}}}$.}
\subsection{Write the above continued fraction as the limit of a sequence. Then write a (first-order) recurrence relation between the terms of the sequence.}
\begin{gather*}
x_1=1+\frac{1}{2}=\frac{3}{2} \\
x_n=1+\frac{1}{1+x_{n-1}}=\frac{2+x_{n-1}}{1+x_{n-1}} \\
\lim_{n\rightarrow\infty}x_n=1+\frac{1}{2+\frac{1}{2+\frac{1}{2+\dotsb}}}
\end{gather*}
\subsection{Show that the sequence is bounded.}
\begin{gather*}
1+\frac{1}{1+x_{n-1}}<1+\frac{1}{1+0} \marginnote{$x_{n-1}\in P$} \\
\therefore \\
\boxed{x_n<2}
\end{gather*}
\subsection{Show that the subsequence of odd-indexed terms and even-indexed terms are monotonic.}
{\allowdisplaybreaks
\begin{align*}
x_{n+2}-x_n&=1+\frac{1}{1+x_{n+1}}-(1+\frac{1}{1+x_{n-1}}) \\
&=\frac{1}{1+x_{n+1}}-\frac{1}{1+x_{n-1}} \\
&=\frac{x_{n-1}-x_{n+1}}{(1+x_{n+1})(1+x_{n-1})} \marginnote{(1)} \\
&=\frac{x_{n-1}}{(1+x_{n+1})(1+x_{n-1})}-\frac{x_{n+1}}{(1+x_{n+1})(1+x_{n-1})} \\
&=\frac{2+x_{n-2}}{(1+x_{n-2})(1+x_{n+1})(1+x_{n-1})}-\frac{2+x_n}{(1+x_n)(1+x_{n+1})(1+x_{n-1})} \\
&=\frac{(2+x_{n-2})(1+x_n)-(2+x_n)(1+x_{n-2})}{(1+x_{n-2})(1+x_n)(1+x_{n+1})(1+x_{n-1})} \\
&=\frac{x_n-x_{n-2}}{(1+x_{n-2})(1+x_n)(1+x_{n+1})(1+x_{n-1})} \marginnote{(2)}
\end{align*}}
\begin{center}
(2) shows $x_n\gtrless x_{n-2}\Rightarrow x_{n+2}\gtrless x_{n}\therefore$ odd and even terms separately monotonic \\
(1) shows if one is monotonic increasing the other is monotonic decreasing
\end{center}
\subsection{Show that the above continued fraction converges and find the limit.}
\begin{gather*}
\text{By MCT, let }\lim_{n\rightarrow\infty}x_{2n+1}=l_1 \land \lim_{n\rightarrow\infty}x_{2n}=l_2 \\
l_1=1+\frac{1}{1+l_2} \land l_2=1+\frac{1}{1+l_1} \\
l_1=l_2=l \\
l=1+\frac{1}{1+l} \\
(l-1)(l+1)=1 \\
l^2-1=1 \\
l^2=2 \\
l=\sqrt{2} \\
\therefore \\
\boxed{\lim_{n\rightarrow\infty}x_n=\sqrt{2}}
\end{gather*}

\end{document}