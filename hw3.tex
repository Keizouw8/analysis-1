\documentclass[letterpaper]{article}
\usepackage{amssymb}
\usepackage{amsmath}
\usepackage{titlesec}
\usepackage{marginnote}
\usepackage{mathtools}
\usepackage{changepage}

\title{Homework 3}
\author{Keizou Wang}
\date{February 19, 2025}

\titleformat{\section}{\normalfont\normalsize}{\bfseries\thesection.}{1em}{}
\titleformat{\subsection}{\normalfont\normalsize}{(\textit{\roman{subsection}})}{1em}{}

\DeclarePairedDelimiter{\ceil}{\lceil}{\rceil}
\DeclarePairedDelimiter\abs{\lvert}{\rvert}

\begin{document}

\maketitle

\section{Use the definition of the limit of a sequence to show that:}
\subsection{$\{\frac{n^2+n+1}{3n^2+1}\}\rightarrow\frac{1}{3}$}
\begin{gather*}
\abs[\bigg]{\frac{n^2+n+1}{3n^2+1}-\frac{1}{3}}<\epsilon \\
\abs[\bigg]{\frac{n+\frac{2}{3}}{3n^2+1}}<\epsilon \\
\frac{n+\frac{2}{3}}{3n^2+1}<\epsilon \marginnote{positive for $n\in\mathbb{N}$} \\
\frac{n+\frac{2}{3}}{\epsilon}<3n^2+1 \\
\frac{1}{\epsilon'}<3n^2+1 \\
\frac{1-\epsilon'}{3\epsilon'}<n^2 \\
n\geq\ceil[\bigg]{\sqrt{\frac{1-\epsilon'}{3\epsilon'}}}=N_\epsilon \\
\exists N_\epsilon:\forall n\geq N_\epsilon\Rightarrow\abs[\bigg]{\frac{n^2+n+1}{3n^2+1}-\frac{1}{3}}<\epsilon \\
\therefore \\
\boxed{\lim_{n\rightarrow\infty}\frac{n^2+n+1}{3n^2+1}=\frac{1}{3}}
\end{gather*}
\subsection{$\{10-\frac{1}{\sqrt{n+\sqrt{n+5}}}\}\rightarrow10$}
\begin{gather*}
\abs[\bigg]{10-\frac{1}{\sqrt{n+\sqrt{n+5}}}-10}<\epsilon \\
\abs[\bigg]{-\frac{1}{\sqrt{n+\sqrt{n+5}}}}<\epsilon \\
\frac{1}{\sqrt{n+\sqrt{n+5}}}<\epsilon \marginnote{always positive} \\
\frac{1}{\sqrt{n+\sqrt{n+5}}}<\frac{1}{\sqrt{n}} \marginnote{smaller denominator} \\
\frac{1}{\sqrt{n}}<\epsilon \\
\sqrt{n}>\frac{1}{\epsilon} \\
n\geq\ceil[\bigg]{\frac{1}{\epsilon^2}}=N_\epsilon \\
\exists N_\epsilon:\forall n\geq N_\epsilon\Rightarrow\abs[\bigg]{10-\frac{1}{\sqrt{n+\sqrt{n+5}}}-10}<\epsilon \\
\therefore \\
\boxed{\lim_{n\rightarrow\infty}10-\frac{1}{\sqrt{n+\sqrt{n+5}}}=10}
\end{gather*}

\section{}
\subsection{Use the definition of the limit of a sequence to show that for a fixed $r$ with $\abs{r}<1$, $\{nr^n\}\rightarrow0$.}
\begin{gather*}
\text{Consider }\lim_{n\rightarrow\infty}\abs[\bigg]{\frac{(n+1)r^{n+1}}{nr^n}}=\abs{r} \\
\Leftarrow \abs[\bigg]{\abs[\bigg]{\frac{(n+1)r^{n+1}}{nr^n}}-\abs{r}}<\epsilon \\
\Leftarrow \abs[\bigg]{\abs[\bigg]{\frac{(n+1)r^{n+1}}{nr^n}}-\abs{r}}\leq\abs[\bigg]{\frac{(n+1)r^{n+1}}{nr^n}-r}<\epsilon \marginnote{$\triangle$-ineq of $<$} \\
\Leftarrow \abs[\bigg]{\frac{(n+1)r}{n}-r}<\epsilon \\
\Leftarrow \abs[\bigg]{r+\frac{r}{n}-r}<\epsilon \\
\Leftarrow \abs[\bigg]{\frac{r}{n}}<\epsilon \\
\Leftarrow n\geq\ceil[\bigg]{\frac{\abs{r}}{\epsilon}}=N_\epsilon \\
\text{Almost-geometric sequence: } \lim_{n\rightarrow\infty}\abs[\bigg]{\frac{x_{n+1}}{x_n}}=r:0<r<1,x_n=nr^n \\
\therefore \\
\boxed{\lim_{n\rightarrow\infty}nr^n=0}
\end{gather*}
\subsection{Use (\textit{i}) to show that $\{\frac{\ln(n)}{n}\}\rightarrow0$.}
\begin{gather*}
\abs[\bigg]{\frac{\ln(n)}{n}-0}<\epsilon \\
\ln(n)<n\epsilon \\
n>\frac{\ln(n)}{\epsilon} \\
n\geq\ceil[\bigg]{\frac{1}{\epsilon'}}=N_\epsilon \marginnote{$\epsilon'$ arbitrarily small still} \\
\exists N_\epsilon:\forall n\geq N_\epsilon\Rightarrow\abs[\bigg]{\frac{\ln(n)}{n}}<\epsilon \\
\therefore \\
\boxed{\lim_{n\rightarrow\infty}\frac{\ln(n)}{n}=0}
\end{gather*}

\section{Find the limits of the following sequences, if they exist:}
\subsection{$x_n=\sqrt{n^2+n}-n$}
\begin{gather*}
\abs{\sqrt{n^2+n}-n-\frac{1}{2}}<\epsilon \\
\abs[\bigg]{\frac{n^2+n-(n+\frac{1}{2})^2}{\sqrt{n^2+n}+(n+\frac{1}{2})}}<\epsilon \\
\abs[\bigg]{\frac{n^2+n-(n^2+n+\frac{1}{4})}{\sqrt{n^2+n}+n+\frac{1}{2}}}<\epsilon \\
\abs[\bigg]{\frac{-\frac{1}{4}}{\sqrt{n^2+n}+n+\frac{1}{2}}}<\epsilon \\
\frac{\frac{1}{4}}{\sqrt{n^2+n}+n+\frac{1}{2}}<\epsilon \\
\Leftarrow\frac{1}{4n}<\epsilon \\
n\geq\ceil[\bigg]{\frac{1}{4\epsilon}}=N_\epsilon \\
\exists N_\epsilon:\forall n\geq N_\epsilon\Rightarrow\abs{\sqrt{n^2+n}-n-\frac{1}{2}}<\epsilon \\
\therefore \\
\boxed{\lim_{n\rightarrow\infty}x_n=\frac{1}{2}}
\end{gather*}
\subsection{$x_n=\frac{1}{(n+1)^2}+\frac{1}{(n+2)^2}+\dotsb+\frac{1}{(n+n)^2}$}
\begin{gather*}
x_n=\sum_{x=1}^{n}\frac{1}{(n+x)^2} \\
0<\sum_{x=1}^{n}\frac{1}{(n+x)^2}<\sum_{x=1}^{n}\frac{1}{x^2}\\
\lim_{n\rightarrow\infty}0=0 \\
\lim_{n\rightarrow\infty}\sum_{x=1}^{n}\frac{1}{x^2}=0 \marginnote{proven in lecture notes} \\
\therefore \\
\boxed{\lim_{n\rightarrow\infty}x_n=0\text{ by squeeze theorem}}
\end{gather*}

\section{Discuss the convergence of the sequence $\frac{1}{n+1}+\frac{1}{n+2}+\dotsb+\frac{1}{n+n}$ by proving:}
\subsection{Show that $\{x_n\}$ is bounded.}
\begin{gather*}
\frac{1}{n+1}+\frac{1}{n+2}+\dotsb+\frac{1}{n+n}=\sum_{x=1}^{n}\frac{1}{n+x} \\
\sum_{x=1}^{n}\frac{1}{n+x}<\sum_{x=1}^{n}\frac{1}{n}=1 \\
\therefore \\
\boxed{\{x_n\}\text{ is upper bounded by }1}
\end{gather*}
\subsection{Show that $\{x_n\}$ is monotonic increasing.}
\begin{gather*}
x_{n+1}-x_n>0 \\
\frac{1}{(n+1)+1}+\frac{1}{(n+1)+2}+\dotsb+\frac{1}{2(n+1)}-(\frac{1}{n+1}+\frac{1}{n+2}+\dotsb+\frac{1}{2n})>0 \\
\frac{1}{n+2}+\frac{1}{n+3}+\dotsb+\frac{1}{2n+2}-(\frac{1}{n+1}+\frac{1}{n+2}+\dotsb+\frac{1}{2n})>0 \\
\frac{1}{2n+1}+\frac{1}{2n+2}-\frac{1}{n+1}>0 \\
\frac{n+1}{4x^3+10x^2+8x+2}>0 \\
n+1>0 \marginnote{true for $n\in\mathbb{N}$} \\
\therefore \\
\boxed{\{x_n\} \text{ is monotonic increasing}}
\end{gather*}
\subsection{Find the limit of $\{x_n\}$ by comparing it to an integral.}
\begin{gather*}
\int_{1}^{2}\frac{1}{x}dx=\lim_{n\rightarrow\infty}\sum_{x=1}^{n}\frac{2-1}{n}\cdot\frac{1}{1+x\frac{2-1}{n}} \\
=\lim_{n\rightarrow\infty}\sum_{x=1}^{n}\frac{1}{n}\cdot\frac{1}{1+\frac{x}{n}} \\
=\lim_{n\rightarrow\infty}\sum_{x=1}^{n}\frac{1}{n+x} \\
\therefore \\
\boxed{\{x_n\}=\int_{1}^{2}\frac{1}{x}dx=\ln(2)}
\end{gather*}

\section{Consider sequence $\{x_n\}$ such that $0\leq x_1<x_2$ and $x_n=\frac{x_{n-1}+x_{n-2}}{2},\forall n\geq3$. Show that $\{x_n\}\rightarrow\frac{x_1+2x_2}{3}$.}
\begin{align*}	
x_n-x_{n-1}&=\frac{x_{n-1}+x_{n-2}}{2}-x_{n-1} \\
&=\frac{x_{n-2}-x_{n-1}}{2} \\
&=-\frac{1}{2}(x_{n-1}-x_{(n-1)-1}) \\
\Delta x_n&=-\frac{\Delta x_{n-1}}{2} \\
&=(x_2-x_1)(-\frac{1}{2})^n \marginnote{recursive to $\Delta x_2=x_2-x_1$} \\
x_n&=x_1+\sum_{k=1}^{n-2}(x_2-x_1)(-\frac{1}{2})^n \\
\lim_{n\rightarrow\infty}x_n&=x_1+\frac{x_2-x_1}{1+\frac{1}{2}} \marginnote{geometric series to $\infty$} \\
&=x_1+\frac{2x_2-2x_1}{3} \\
&=\boxed{\frac{x_1+2x_2}{3}}
\end{align*}

\end{document}