\documentclass[letterpaper]{article}
\usepackage{amssymb}
\usepackage{amsmath}
\usepackage{titlesec}
\usepackage{marginnote}
\usepackage{mathtools}

\title{Homework 1}
\author{Keizou Wang}
\date{February 5, 2025}

\titleformat{\section}{\normalfont\normalsize}{\bfseries\thesection.}{1em}{}
\titleformat{\subsection}{\normalfont\normalsize}{(\textit{\roman{subsection}})}{1em}{}

\begin{document}

\maketitle

\section{Let $\mathbb{F}$ be an ordered field and $a, b, \epsilon \in \mathbb{F}$.}
\subsection{Show that if $a < b + \epsilon$ for every $\epsilon > 0$, then $a \leq b$.}
\begin{gather*}
\mathbb{F}
\text{ is ordered} \Leftrightarrow
P \hspace{3pt} \exists \hspace{3pt} \mathbb{F}
\text{ where $P$ is the positive set and } \epsilon \in P \marginnote{order axiom} \\
a<b+\epsilon \\
a-b<\epsilon \marginnote{$<$ additivity} \\
a-b<\epsilon \Rightarrow a-b\neq\epsilon \\
a-b\neq\epsilon \Leftrightarrow a-b\notin P \\
a-b\notin P \Rightarrow a-b=0 \lor -(a-b) \in P \marginnote{trichotomy} \\
b-a=0 \lor b-a \in P \\
\therefore \\
\boxed{a \leq b}
\end{gather*}
\subsection{Use (\textit{i}) to show that if $|a-b|<\epsilon$ for all $\epsilon>0$, then $a=b$.\protect\footnote{Proof using trichotomy in notes}}
\begin{gather*}
|a-b|<\epsilon \\
-\epsilon < a-b < \epsilon \marginnote{FT abs-value} \\
-\epsilon < a-b \land a-b < \epsilon \\
b<a+\epsilon \land a<b+\epsilon \\
b\leq a \land a\leq b \\
\therefore \\
\boxed{a = b}
\end{gather*}

\section{Let $A \subseteq \mathbb{R}$. Define $-A=\{-a:a\in A\}$. Suppose that $A$ is non-empty and bounded below. Show that
$\inf(A)=-\sup(-A)$.}
\begin{gather*}
x \in A \Rightarrow x \geq \inf(A) \marginnote{inf analytic definition lower bound} \\
-x\leq-\inf(A) \\
x'\in -A \Rightarrow x'=-x,x\in A \Rightarrow x' \leq-\inf(A) \\
-\inf(A) \text{ is an upper bound of $-A$} \\ 
\text{Given } \epsilon>0,\hspace{3pt}\exists\hspace{3pt}x\in A:x<\inf(A)+\epsilon \\
-x > -\inf(A) - \epsilon \\
\hspace{3pt}\exists\hspace{3pt}k\in -A:k=-x,k>-\inf(A)-\epsilon \marginnote{satisfies both requirements for sup($-A$)} \\
\therefore \\
\sup(-A)=-\inf(A) \\
\boxed{\inf(A)=-\sup(-A)}
\end{gather*}

\section{Let $A=\{\frac{n}{n+1}:n\in\mathbb{N}\}$. Prove that $\sup(A)=1, \inf(A)=\frac{1}{2}$.}
\begin{gather*}
A=\{f(n):n\in\mathbb{N},f(x)=\frac{x}{x+1}\} \\
\line(1, 0){125} \\
\frac{n}{n+1}\geq\frac{1}{2} \\
2n\geq n+1 \\
n \geq 1 \Rightarrow \frac{1}{2} \text{ is a lower bound of } A \marginnote{$n\geq1$ is valid by definition} \\
f(1)=\frac{1}{2} \\
\frac{1}{2} - f(1) = 0 \\
\epsilon + \frac{1}{2} - f(1) = \epsilon \marginnote{where $\epsilon>0$} \\
(\frac{1}{2}+\epsilon) - f(1) > 0 \\
f(1)<\frac{1}{2}+\epsilon \\
\text{Given } \epsilon>0, \hspace{3pt} \exists \hspace{3pt} x \in A:x<\frac{1}{2}+\epsilon \marginnote{when $x=f(1)$} \\
\therefore \\
\boxed{\inf(A)=\frac{1}{2}} \\
\line(1, 0){125}
\end{gather*}
\begin{gather*}
\frac{n}{n+1}\leq1 \\
n\leq n+1 \\
0\leq1 \Rightarrow 1 \text{ is an upper bound of } A \\
\frac{n}{n+1} < 1-\epsilon \marginnote{where $\epsilon>0$} \\
\epsilon<1-\frac{n}{n+1} \\
0<\epsilon<\frac{1}{n+1} \\
\frac{1}{n+1}>0 \marginnote{true for $n \in \mathbb{N}$} \\
\text{Given } \epsilon>0: \hspace{3pt} \exists \hspace{3pt} x \in A:x>1-\epsilon  \\
\therefore \\
\boxed{\sup(A)=1}
\end{gather*}

\section{Let $A,B\supseteq\mathbb{R}:$}
\subsection{$\exists\sup(A),\exists\sup(B),A\subseteq B$. Show that $\sup(A)\leq\sup(B)$.}
\begin{gather*}
\exists b\in B: b > \sup(A) \lor \neg \exists b\in B: b > \sup(A) \marginnote{a tautology ($a\lor\neg a$)} \\
\line(1, 0){125} \\
\text{Case 1} \\
\exists b\in B: b > \sup(A) \\
\sup(A) < b \leq \sup(B) \\
\sup(A) < \sup(B) \\
\line(1, 0){125} \\
\text{Case 2} \\
\neg \exists b\in B: b > \sup(A) \\
\Leftrightarrow b\in B\Rightarrow b\leq\sup(A) \\
\Rightarrow \text{$\sup(A)$ is an upperbound of $B$} \\
\text{Given }\epsilon>0: \exists a \in A: a>\sup(A)-\epsilon \\
B\supset A \Rightarrow a\in B \\
a \in B: a>\sup(A)-\epsilon \\
\sup(A) = \sup(B) \\
\line(1, 0){125} \\
\sup(A) < \sup(B) \lor \sup(A) = \sup(B) \\
\therefore \\
\boxed{\sup(A)\leq\sup(B)}
\end{gather*}
\subsection{$\sup(A)<\sup(B)$. Show that there exists $b\in B$ that is an upper bound of $A$. Show that this result does not hold if we instead assume that $\sup(A)\leq\sup(B)$.}
\begin{gather*}
\sup(A)\leq\sup(B) \\
\Diamond(\sup(A)=\sup(B)=k,k\in\mathbb{R}) \\
b\in B \Rightarrow b\leq k \\
a\in A \Rightarrow a\leq k \\
\Diamond(\max(b)<k\land \max(a)=k) \\
\max(b)<\max(a) \\
\therefore \\
\boxed{\sup(A)\leq\sup(B) \Rightarrow \Diamond\neg\exists b\in B:b\text{ is an upper bound of }A} \\
\line(1, 0){125} \\
\sup(A)<\sup(B) \\
\text{Given }\epsilon>0:\exists b\in B:b>\sup(B)-\epsilon \\
b+\epsilon>\sup(B) \Rightarrow b+\epsilon>\sup(A) \\
b+\epsilon-\sup(A)>0 \\
b-\sup(A)>-\epsilon \\
-(b-\sup(A))<\epsilon \\
-(b-\sup(A))\notin P \\
b-\sup(A)\in P \lor b-\sup(A)=0 \marginnote{trichotomy} \\
b\geq\sup(A)\geq a\\
\therefore \\
\boxed{\sup(A)<\sup(B) \Rightarrow \exists b\in B:b\text{ is an upper bound of }A}
\end{gather*}

\section{For $A,B\supseteq \mathbb{R}$, define}
\begin{gather*}
A+B=\{a+b:a\in A,b\in B\} \\
A\cdot B=\{a\cdot b:a\in A,b\in B\}
\end{gather*}
\subsection{Determine $\{3,1,0\}+\{2,0,2,3\}$ and $\{3,1,0\}\cdot\{2,0,2,3\}$.}
\begin{align*}
A&=\{3,1,0\} \\
B&=\{2,0,2,3\} \\
A+B&=\{3+2,3+0,3+2,3+3,1+2,1+0,1+2,1+3,0+2,0+0,0+2,0+3\} \\
&=\{5,3,6,1,4,2,0\} \\
A\cdot B&= \{3\cdot 2,3\cdot 0,3\cdot 2,3\cdot 3,1\cdot 2,1\cdot 0,1\cdot 2,1\cdot 3,0\cdot 2,0\cdot 0,0\cdot 2,0\cdot 3\} \\
&=\{6,0,9,2,3\}
\end{align*}
\subsection{Assume that $\sup(A)$ and $\sup(B)$ exist. Prove that $\sup(A+B)=\sup(A)+\sup(B)$.}
\begin{gather*}
a \in A \Rightarrow a \leq \sup(A) \\
b \in B \Rightarrow b \leq \sup(B) \\
a+b\leq\sup(A)+\sup(B) \\
c \in A+B \Rightarrow c=a+b,a\in A,b\in B \\
c\leq\sup(A)+\sup(B) \\
\sup(A)+\sup(B)\text{ is an upper bound of }A+B \\
\text{Given }\epsilon>0: \\
\exists a\in A:a>\sup(A)-\epsilon \\
\exists b\in B:b>\sup(B)-\epsilon \\
a+b>\sup(A)+\sup(B)-2\epsilon \\
2\epsilon>0 \Rightarrow a+b>\sup(A)+\sup(B)-\epsilon \\
a+b\in A+B \Rightarrow \exists c \in A+B:c>(\sup(A)+\sup(B))-\epsilon \\
\therefore \\
\boxed{\sup(A+B)=\sup(A)+\sup(B)}
\end{gather*}
\subsection{Give an example of sets $A,B$ where $\sup(A\cdot B)\neq\sup(A)\cdot\sup(B)$.}
\begin{gather*}
A=\{-1\}, B=\{1,2\} \\
\sup(A)=-1 \\
\sup(B)=2 \\
\sup(A)\cdot\sup(B)=-2 \\
A\cdot B=\{-1, -2\} \\
\sup(A\cdot B)=-1 \\
-1\neq-2 \\
\boxed{\sup(A\cdot B)\neq\sup(A)\cdot\sup(B)}
\end{gather*}
\pagebreak
\setcounter{section}{0}

\begin{center}\Large{Notes}\end{center}

\section{Question 1-ii proof by trichotomy}
\begin{gather*}
|a-b|<\epsilon \\
-\epsilon < a-b < \epsilon \marginnote{FT abs-value} \\
-\epsilon < a-b \land a-b < \epsilon \\
\epsilon > -(a-b) \land a-b < \epsilon \marginnote{$<$ multiplicity} \\
-(a-b) \notin P \land a-b \notin P \\
\Rightarrow a-b=0 \marginnote{trichotomy} \\
\therefore \\
\boxed{a = b}
\end{gather*}


\end{document}