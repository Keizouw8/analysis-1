\documentclass[letterpaper]{article}
\usepackage{amssymb}
\usepackage{amsmath}
\usepackage{titlesec}
\usepackage{marginnote}
\usepackage{mathtools}
\usepackage{changepage}

\title{Homework 5}
\author{Keizou Wang}
\date{March 7, 2025}

\titleformat{\section}{\normalfont\normalsize}{\bfseries\thesection.}{1em}{}
\titleformat{\subsection}{\normalfont\normalsize}{(\textit{\roman{subsection}})}{1em}{}

\DeclarePairedDelimiter{\ceil}{\lceil}{\rceil}
\DeclarePairedDelimiter{\abs}{\lvert}{\rvert}
\DeclarePairedDelimiter{\paren}{(}{)}

\begin{document}

\maketitle

\section{Let $\sum x_n$ be an infinite series of positive terms. (\textbf{Do any three})}
\subsection{Show that $\sum x_n$ is convergent $\Rightarrow\sum x_n^2$ is convergent.}
\begin{gather*}
\sum x_n=l\Rightarrow \lim_{n\rightarrow\infty}x_n=0 \\
\exists N_\epsilon:n>N_\epsilon\Rightarrow x_n<\epsilon \marginnote{$x_n\in P$ so $\abs{x_n}$ not needed} \\
\exists N_1:n>N_1\Rightarrow x_n<1 \\
\sum x_n^2\Leftrightarrow\sum_{n=N_1}^{\infty}x_n^2 \\
n>N_1:x_n^2<x_n\land\sum_{n=N_1}^{\infty}x_n\text{ is convergent} \\
\sum_{n=N_1}^{\infty}x_n^2\text{ is convergent} \\
\therefore \\
\boxed{\sum x_n^2\text{ is convergent}}
\end{gather*}
\subsection{Show that $\sum x_n$ is convergent $\Rightarrow\sum \frac{x_n}{n}$ is convergent.}
\begin{gather*}
\frac{x_n}{n}<x_n \land \sum x_n \text{ is convergent} \\
\therefore \\
\boxed{\sum\frac{x_n}{n}\text{ is convergent}}
\end{gather*}
\subsection{Show that $\sum x_n$ is convergent $\Rightarrow\sum \sqrt{x_nx_{n+1}}$ is convergent.}
\begin{center}\end{center}
\subsection{Show that $\sum x_n$ is convergent $\Rightarrow\sum \frac{x_n}{1+x_n}$ is convergent.}
\begin{gather*}
\frac{x_n}{1+x_n}<x_n \land \sum x_n \text{ is convergent} \\
\therefore \\
\boxed{\sum\frac{x_n}{1+x_n}\text{ is convergent}}
\end{gather*}
\subsection{Let $y_n=\frac{x_1+x_2+\dotsb+x_n}{n}$. Show that $\sum y_n$ is divergent.}

\section{Test the convergence of the following series (\textbf{Do any three}):}
\subsection{$\frac{1}{1+2^{-1}}+\frac{1}{1+2^{-2}}+\frac{1}{1+2^{-3}}+\dotsb$}
\begin{gather*}
\frac{1}{1+2^{-1}}+\frac{1}{1+2^{-2}}+\frac{1}{1+2^{-3}}+\dotsb=\sum x_n:x_n=\frac{1}{1+2^{-n}} \\
\lim_{n\rightarrow\infty}x_n=1\neq0 \\
\therefore \\
\boxed{\frac{1}{1+2^{-1}}+\frac{1}{1+2^{-2}}+\frac{1}{1+2^{-3}}+\dotsb\text{ is divergent}}
\end{gather*}
\subsection{$\frac{\ln1}{1^p}+\frac{\ln2}{2^p}+\frac{\ln3}{3^p}+\frac{\ln4}{4^p}+\dotsb;p\in\mathbb{R}$}
\subsection{$1+\frac{2^2}{2!}+\frac{3^2}{3!}+\frac{4^2}{4!}+\dotsb$}
\begin{gather*}
x_n=\frac{n^2}{n!} \\
\begin{aligned}	
\frac{x_{n+1}}{x_n}&=\frac{(n+1)^2}{(n+1)!}\cdot\frac{n!}{n^2} \marginnote{$\forall x_n\rightarrow x_n\in P$} \\
&=\frac{(n+1)^2}{n^2(n+1)} \\
&=\frac{n+1}{n^2} \rightarrow 0
\end{aligned} \\
0<1 \\
\therefore \\
\boxed{1+\frac{2^2}{2!}+\frac{3^2}{3!}+\frac{4^2}{4!}+\dotsb\text{ is convergent}}
\end{gather*}
\subsection{$\frac{1^p}{p^1}+\frac{2^p}{p^2}+\frac{3^p}{p^3}+\frac{4^p}{p^4}+\dotsb;p\in\mathbb{R}$}
\subsection{$1+\frac{x}{1}+\frac{x^2}{2}+\frac{x^3}{3}+\dotsb;x\in\mathbb{R}$}
\begin{gather*}
y_n=\frac{x^n}{n} \\
\line(1, 0){125} \\
\text{Case 1: } x=0 \\
y_n=0 \\
\sum y_n=0 \\
\sum y_n \text{ is convergent} \\
\line(1, 0){125} \\
\text{Case 2: } x\in P \\
\begin{aligned}
\frac{y_{n+1}}{y_n}&=\frac{x^{n+1}}{n+1}\cdot\frac{n}{x^n} \\
&=x\cdot\frac{n}{n+1}\rightarrow x
\end{aligned} \\
\sum y_n \text{ is convergent if } x<1 \marginnote{$x=1$ is harmonic} \\
\line(1, 0){125} \\
\text{Case 3: } -x\in P \\
\text{Let }k=\abs{x} \\
\begin{aligned}
y_n&=(-1)^n\frac{k^n}{n} \\
&=(-1)^nz_n
\end{aligned} \\
z_n=\frac{k^n}{n} \\
z_{n+1}\leq z_n \marginnote{when $z_n$ m.d. for AST} \\
\begin{aligned}
&\Leftarrow\frac{k^{n+1}}{n+1}\leq\frac{k^n}{n} \\
&\Leftarrow n\cdot k^{n+1}\leq(n+1)k^n \\
&\Leftarrow nk\leq n+1 \\
&\Leftarrow k\leq \frac{n+1}{n}\rightarrow1
\end{aligned} \\
\sum y_n \text{ is convergent if } x\geq-1 \\
\line(1, 0){125} \\
\therefore \\
\boxed{\sum y_n \text{ is convergent if } -1\leq x<1}
\end{gather*}

\section{}
\subsection{Let $\{x_n\}$ be a m.d. sequence. Prove that: $\sum x_n \text{ converges} \Rightarrow \{nx_n\}\rightarrow0$.}
\begin{gather*}
\text{Given }\epsilon>0 \\
\exists N_\epsilon:\forall n>N_\epsilon\Rightarrow|x_n|<\epsilon \\
\exists N:\forall n>N,\forall p\in\mathbb{N}\Rightarrow\abs{x_{n+1}+x_{n+2}+\dotsb+x_{n+p}}<\epsilon \\
\abs[\bigg]{\sum_{k=1}^{p}x_{n+k}}<\epsilon \\
p\abs{x_n}<\epsilon \marginnote{$x_n$ m.d.} \\
\abs{x_n}<\frac{\epsilon}{p} \\
\begin{aligned}
&\abs{nx_n}<\epsilon \\
&\Leftarrow \frac{n\epsilon}{p}<\epsilon \\
&\Leftarrow p>n \marginnote{p over all $\mathbb{N}$, can $>n$}
\end{aligned} \\
\therefore \\
\boxed{\lim_{n\rightarrow\infty}nx_n=0}
\end{gather*}
\subsection{Show that the converse of (\textit{i}) does not hold.}
\begin{gather*}
\boxed{x_n=\frac{1}{n\ln(n)}}
\end{gather*}
\subsection{Show that (\textit{i}) does not hold if \textit{monotonic decreasing} is removed.}
\begin{gather*}
\boxed{x_n=\frac{(-1)^n}{n}} \\
\because \\
\sum x_n\text{ is convergent} \marginnote{\textbf{2.}(\textit{v})} \\
\lim_{n\rightarrow\infty} nx_n\text{ diverges}
\end{gather*}

\section{\textbf{Cauchy’s Condensation Test} says that if $f(n)$ is a monotone decreasing function of positive values,
then $\sum f(n)$ is convergent $\Leftrightarrow\sum2^nf(2^n)$ is convergent.}
\subsection{Use Cauchy Condensation Test to discuss the convergence of $\sum\frac{1}{n^p}$ for all values of $p$.}
\begin{gather*}
\begin{aligned}	
\sum\frac{1}{n^p}\text{ converges}&\Leftrightarrow\sum\frac{2^n}{2^{np}}\text{ converges} \\
&\Leftarrow\sum2^{n-np} \\
&\Leftarrow\sum\paren[\bigg]{\frac{1}{2}}^{n(p-1)} \\
&\Leftarrow\sum\paren[\bigg]{\frac{1}{2}^{p-1}}^n \\
&\Leftarrow \frac{1}{2}^{p-1}<1 \marginnote{$r$ of geo. series} \\
&\Leftarrow p-1>0 \\
&\Leftarrow p>1 \\
\end{aligned} \\
\therefore \\
\boxed{\sum\frac{1}{n^p} \text{ is convergent if }p>1}
\end{gather*}
\subsection{Use Cauchy Condensation Test to discuss the convergence of $\sum\frac{1}{n(\ln n)^p}$ for all values of $p$.}
\begin{gather*}
\begin{aligned}
\sum\frac{1}{n(\ln n)^p}\text{ converges}&\Leftrightarrow\sum\frac{2^n}{2^n(\ln(2^n))^p}\text{ converges} \\
&\Leftarrow\sum\frac{1}{(n\ln2)^p} \\
&\Leftarrow\ln^{-p}(2)+\sum\frac{1}{n^p} \\
&\Leftarrow\sum\frac{1}{n^p} \marginnote{p-series}
\end{aligned} \\
\therefore \\
\boxed{\sum\frac{1}{n(\ln n)^p}\text{ is convergent if }p>1}
\end{gather*}

\section{}
\subsection{Show that if $\sum x_n$ is abs. convergent and $\{y_n\}$ is a bounded sequence, then $\sum x_ny_n$ is abs. convergent.}
\begin{gather*}
\exists\alpha:\forall y_n\Rightarrow y_n\leq\alpha \\
\sum\abs{x_ny_n}\leq\sum\abs{x_n\alpha}=\alpha\sum\abs{x_n} \marginnote{constant times conv. series} \\
\therefore \\
\boxed{\sum\abs{x_ny_n}\text{ is abs. convergent}}
\end{gather*}
\subsection{Use (\textit{i}) to discuss the convergence of $\frac{2}{1^3}-\frac{3}{2^3}+\frac{4}{3^3}-\frac{5}{4^3}+\dotsb$.}
\begin{gather*}
\frac{2}{1^3}-\frac{3}{2^3}+\frac{4}{3^3}-\frac{5}{4^3}+\dotsb=\sum\frac{1}{n^2}\cdot\frac{n+1}{n} \\
x_n=\frac{1}{n^2} \text{ is abs. convergent} \marginnote{p-series} \\
y_n=\frac{n+1}{n} \text{ bounded within } (1, 2] \\
\therefore \\
\boxed{\frac{2}{1^3}-\frac{3}{2^3}+\frac{4}{3^3}-\frac{5}{4^3}+\dotsb\text{ is abs. convergent}}
\end{gather*}
\subsection{Use (\textit{i}) to show that if both $\sum x_n$ and $\sum y_n$ are abs. convergent, then $\sum x_ny_n$ is abs. convergent.}
\begin{gather*}
\sum y_n\text{ abs. convergent} \Rightarrow \abs{y_n}\text{ bounded} \\
\exists\alpha:\forall y_n\Rightarrow\abs{y_n}\leq\alpha \\
\sum\abs{x_ny_n}\leq\alpha\sum\abs{x_n} \\
\therefore \\
\boxed{\sum x_ny_n\text{ is abs. convergent}}
\end{gather*}
\subsection{Do (\textit{i}) and (\textit{iii}) still hold if we replace \textit{abs. convergent} with \textit{convergent}?}
\begin{gather*}
\text{For part (\textit{i}): Consider }x_n=\sum\frac{(-1)^n}{n},y_n=(-1)^n \\
x_n\text{ is convergent and }y_n\text{ is bounded between [-1, 1]} \\
\sum x_ny_n\text{ diverges} \marginnote{harmonic} \\
\therefore \\
\boxed{\text{(\textit{i}) does not hold}} \\
\line(1, 0){125} \\
\text{For part (\textit{iii}): } \exists\alpha,\beta:\forall y_n\Rightarrow\beta\leq y_n\leq\alpha \\
\exists a\in\{\alpha,\beta\}:\sum x_ny_n\leq \alpha\sum x_n \\
\alpha\sum x_n \text{ is convergent} \\
\therefore \\
\boxed{\sum x_ny_n\text{ is convergent}}
\end{gather*}
\end{document}