\documentclass[letterpaper]{article}
\usepackage{xcolor}
\usepackage{amssymb}
\usepackage{amsmath}
\usepackage{titlesec}
\usepackage{marginnote}
\usepackage{mathtools}
\usepackage{changepage}

\title{Homework 7}
\author{Keizou Wang}
\date{April 14, 2025}

\titleformat{\section}{\normalfont\normalsize}{\bfseries\thesection.}{1em}{}
\titleformat{\subsection}{\normalfont\normalsize}{(\textit{\roman{subsection}})}{1em}{}

\DeclarePairedDelimiter{\ceil}{\lceil}{\rceil}
\DeclarePairedDelimiter{\floor}{\lfloor}{\rfloor}
\DeclarePairedDelimiter\abs{\lvert}{\rvert}
\DeclarePairedDelimiter\range{[}{]}
\DeclarePairedDelimiter\paren{(}{)}

\begin{document}

\maketitle

\section{Use the definition of derivative, knowledge of trigonometry and standard limits to find the derivatives of:}
\subsection{$\sin(x)$}
\begin{gather*}
	\begin{aligned}
		\sin'(x)&=\lim_{h\rightarrow 0}\frac{\sin(x+h)-\sin(x)}{h} \\
		&=\lim_{h\rightarrow 0}\frac{\sin(x)\cos(h)+\sin(h)\cos(x)-\sin(x)}{h} \marginnote{angle sum formula} \\
		&=\lim_{h\rightarrow 0}\frac{\sin(x)(\cos(h)-1)+\sin(h)\cos(x)}{h} \\
		&=\lim_{h\rightarrow 0}\sin(x)\frac{\cos(h)-1}{h}+\cos(x)\frac{\sin(h)}{h} \\
		&=\sin(x)\lim_{h\rightarrow 0}\frac{\cos(h)-1}{h}+\cos(x)\lim_{h\rightarrow 0}\frac{\sin(h)}{h} \\
	\end{aligned} \\
	\lim_{h\rightarrow 0}\frac{\sin(h)}{h}=1 \marginnote{standard limit} \\
	\begin{aligned}
		\lim_{h\rightarrow 0}\frac{\cos(h)-1}{h}&=\lim_{h\rightarrow 0}\frac{\cos(h)-1}{h}\cdot\frac{\cos(h)+1}{\cos(h)+1} \\
		&=\lim_{h\rightarrow 0}\frac{\cos^2(h)-1}{h(\cos(h)+1)} \\
		&=\lim_{h\rightarrow 0}-\frac{\sin^2(h)}{h(\cos(h)+1)} \\
		&=-\lim_{h\rightarrow 0}\frac{\sin(h)}{h}\cdot\frac{\sin(h)}{\cos(h)+1} \\
		&=-\lim_{h\rightarrow 0}\frac{\sin(h)}{h}\cdot\lim_{h\rightarrow 0}\frac{\sin(h)}{\cos(h)+1} \\
		&=0
	\end{aligned} \\
	\sin'(x)=\sin(x)\cdot0+\cos(x)\cdot1 \\
	\therefore \\
	\boxed{\sin'(x)=\cos(x)}
\end{gather*}
\subsection{$\cos(x)$}
\begin{gather*}
	\begin{aligned}
		\cos'(x)&=\lim_{h\rightarrow0}\frac{\cos(x+h)-\cos(x)}{h} \\
		&=\lim_{h\rightarrow0}\frac{\cos(x)\cos(h)-\sin(x)\sin(h)-\cos(x)}{h} \\
		&=\lim_{h\rightarrow0}\frac{\cos(x)(\cos(h)-1)-\sin(x)\sin(h)}{h} \\
		&=\cos(x)\lim_{h\rightarrow0}\frac{\cos(h)-1}{h}-\sin(x)\lim_{h\rightarrow0}\frac{\sin(h)}{h} \marginnote{$\lim_{h\rightarrow0}\frac{\cos(h)-1}{h}$ found in part i} \\
		&=\cos(x)\cdot0-\sin(x)\cdot1 
	\end{aligned} \\
	\therefore \\
	\boxed{\cos'(x)=-\sin(x)}
\end{gather*}
\subsection{$\tan(x)$}
\begin{gather*}
	\begin{aligned}
		\tan'(x)&=\lim_{h\rightarrow0}\frac{\tan(x+h)-\tan(x)}{h} \\
		&=\lim_{h\rightarrow0}\frac{1}{h}\paren[\bigg]{\frac{\tan(x)+\tan(h)}{1-\tan(x)\tan(h)}-\tan(x)} \\
		&=\lim_{h\rightarrow0}\frac{1}{h}\cdot\frac{\tan(x)+\tan(h)-\tan(x)+\tan^2(x)\tan(h)}{1-\tan(x)\tan(h)} \\
		&=\lim_{h\rightarrow0}\frac{1}{h}\cdot\frac{\tan(h)(1+\tan^2(x))}{1-\tan(x)\tan(h)} \\
		&=\lim_{h\rightarrow0}\frac{1}{h}\cdot\frac{\tan(h)\sec^2(x)}{1-\tan(x)\tan(h)} \\
		&=\sec^2(x)\lim_{h\rightarrow0}\frac{1}{h}\cdot\frac{\tan(h)}{1-\tan(x)\tan(h)} \\
		&=\sec^2(x)\lim_{h\rightarrow0}\frac{\sin(h)}{h}\cdot\frac{1}{\cos(h)}\cdot\frac{1}{1-\tan(x)\tan(h)} \\
		&=\sec^2(x)\lim_{h\rightarrow0}\frac{\sin(h)}{h}\cdot\frac{1}{\cos(h)}\cdot\frac{1}{1-\tan(x)\tan(h)} \\
		&=\sec^2(x)\lim_{h\rightarrow0}\frac{\sin(h)}{h}\cdot\lim_{h\rightarrow0}\frac{1}{\cos(h)}\cdot\lim_{h\rightarrow0}\frac{1}{1-\tan(x)\tan(h)} \\
		&=\sec^2(x)\cdot1\cdot1\cdot1 \\
	\end{aligned} \\
	\therefore \\
	\boxed{\tan'(x)=\sec^2(x)}
\end{gather*}

\pagebreak

\section{[\textcolor{red}{Continuous almost everywhere, but differentiable nowhere}] Prove that \textit{Thomae's function} $f:[0,1]\rightarrow[0,1]$ is not differentiable at any point.}
\begin{displaymath}
	f(x) = \left\{
		\begin{array}{ll}
			\frac{1}{n} & x=\frac{m}{n}; m,n\in\mathbb{N};n>0;\text{ in lowest terms} \\
			0 & x=0 \text{ and otherwise}
		\end{array}
	\right.
\end{displaymath}
\begin{gather*}
	\text{Case 1: } x\in\mathbb{Q} \\
	f \text{ disc. @ } x \marginnote{shown in hw6} \\
	\therefore \\
	f \text{ not diff. @ } x \\
	\line(1, 0){125} \\
	\text{Case 2: } x\notin\mathbb{Q} \\
	\text{Assume }f\text{ diff. @ }x \\
	\text{Let }g(a)=\frac{f(a)-f(x)}{a-x}=\frac{f(a)}{a-x} \\
	\begin{aligned}
		\exists f'&\Rightarrow\exists L:\lim_{a\rightarrow x}g=L \\
		&\Rightarrow \forall\epsilon>0, \exists\delta:\forall a:|a-x|<\delta,|g(a)-L|<\epsilon \\
	\end{aligned} \\
	\mathbb{Q}\text{ dense in }[0,1]\Rightarrow\exists a=\frac{m}{n}:m,n\in\mathbb{N},|a-x|<\delta \\
	\text{Consider }a:n>\frac{1}{\delta(\epsilon+|L|)} \\
	|g(a)-L|<\epsilon \\
	\begin{aligned}
		&\Leftarrow\abs[\bigg]{\frac{f(a)}{a-x}-L}<\epsilon \\
		&\Leftarrow\abs[\bigg]{\frac{\frac{1}{n}}{a-x}}-\abs{L}<\epsilon \\
		&\Leftarrow\abs[\bigg]{\frac{1}{n(a-x)}}<\epsilon+\abs{L} \\
		&\Leftarrow\frac{1}{n\delta}<\epsilon+\abs{L} \\
		&\Leftarrow\frac{1}{\frac{1}{\delta(\epsilon+|L|)}\delta}<\epsilon+\abs{L} \\
		&\Leftarrow\epsilon+\abs{L}<\epsilon+\abs{L} \\
	\end{aligned} \\
	\text{Contradiction} \\
	\therefore \\
	f\text{ not diff. @ }x \\
	\line(1, 0){125} \\
	\boxed{f\text{ diff. nowhere}}
\end{gather*}

\section{Which is greater, $e^\pi$ or $\pi^e$?}
\begin{gather*}
	\begin{aligned}
		\text{Let } &f(x)=\ln(x), f'(x)=\frac{1}{x} \\
		& g(x)=x, g'(x)=1
	\end{aligned} \\
	\exists c\in(\pi^e,e^\pi):\frac{f'(c)}{g'(c)}=\frac{f(e^\pi)-f(\pi^e)}{g(e^\pi)-g(\pi^e)} \\
	f'(c)(g(e^\pi)-g(\pi^e))=g'(c)(f(e^\pi)-f(\pi^e)) \\
	\frac{e^\pi-\pi^e}{c}=\ln(e^\pi)-\ln(\pi^e) \\
	\frac{e^\pi-\pi^e}{c}=\pi-e\ln(\pi) \\
	e^\pi-\pi^e>\pi-e\ln(\pi) \\
	\line(1, 0){125} \\
	\text{Consider }h(x)=x-e\ln(x),h'(x)=1-\frac{e}{x} \\
	h(e)=e-e\ln(e)=0 \\
	\forall x>e, \frac{e}{x}<1\Rightarrow h'(x)>0 \\
	\pi>e\Rightarrow h(\pi)>0 \\
	\line(1, 0){125} \\
	e^\pi-\pi^e>0 \\
	\therefore \\
	\boxed{e^\pi>\pi^e}
\end{gather*}

\section{Prove that in any interval in which the functions $f,g,f',g'$ are continuous and $fg'-f'g\neq0$, then the roots of f and g `separate' each other.}
\begin{gather*}
	\begin{aligned}
		\text{Let } &F(x)=f(x)g'(x)-f'(x)g(x) \\
		&R=\{x:f(x)=0\} \marginnote{roots of $f$} \\
		&h(x)=\frac{f(x)}{g(x)} \\
		&\begin{aligned}
			h'(x)&=\frac{f'(x)g(x)-f(x)g'(x)}{g^2(x)} \\
			&=\frac{-F(x)}{g^2(x)} \\
		\end{aligned} \\
		&a,b\in R:a<b,(a,b)\cap R=\emptyset \marginnote{$a,b$ are consecutive roots of $f$}
	\end{aligned} \\
	\line(1, 0){125} \\
	\text{Assume }\nexists c\in(a,b):g(c)=0 \\
	g(x)\neq0, \forall x\in(a,b) \Rightarrow h(x) \text{ diff. @ }(a,b) \\
	F(x)\neq,\forall x\Rightarrow h'(x)\neq0,\forall x \\
	h(a)=h(b)=0\land h\text{ diff. @ }(a,b)\Rightarrow\exists c\in(a,b):h'(c)=0 \marginnote{Rolle's thm.} \\
	h'(x)\neq0,\forall x \land \exists c\in(a,b):h'(c)=0\Rightarrow\text{ Contradiction} \\
	\therefore \\
	\boxed{\exists c\in(a,b):g(c)=0}
\end{gather*}

\section{[\textcolor{red}{Some inequalities using MVT}] Apply MVT to prove the following inequalities}
\subsection{$x-\frac{x^3}{6}<\sin(x)<x;\forall0<x<\frac{\pi}{2}$}
\begin{gather*}
	\sin'(x)=\cos(x) \\
	\forall x, \exists c\in(0,x):\sin'(c)=\frac{\sin(x)-\sin(0)}{x-0} \\
	\cos(c)=\frac{\sin(x)}{x} \\
	x\cos(c)=\sin(x) \\
	\forall x\in\paren[\bigg]{0,\frac{\pi}{2}},\cos(x)<1\Rightarrow x>\sin(x) \\
	\line(1, 0){125} \\
	f(x)=\sin(x)-x+\frac{x^3}{6} \displaybreak \\
	f'(x)=\cos(x)-1+\frac{x^2}{2} \\
	\forall x, \exists c\in(0,x):f'(c)=\frac{f(x)-f(0)}{x-0} \\
	\frac{\sin(x)-x+\frac{x^3}{6}}{x}=\cos(c)-1+\frac{c^2}{2} \\
	\sin(x)-x+\frac{x^3}{6}=x\paren[\bigg]{\cos(c)+\frac{c^2}{2}}-x \\
	\sin(x)+\frac{x^3}{6}=x\paren[\bigg]{\cos(c)+\frac{c^2}{2}} \\
	\text{Let }g(x)=\cos(x)+\frac{x^2}{2} \\
	g(0)=1, g'(x)=x-\sin(x) \\
	\begin{aligned}
		\forall x\in(0,c), \sin(x)<x &\Rightarrow \forall x\in(0,c), g'(x) > 0 \\
		&\Rightarrow g(c)>1
	\end{aligned} \\
	\sin(x)+\frac{x^3}{6}>x \\
	\sin(x)>x-\frac{x^3}{6} \\
	\line(1, 0){125} \\
	\therefore \\
	\boxed{x-\frac{x^3}{6}<\sin(x)<x;\forall0<x<\frac{\pi}{2}}
\end{gather*}
\subsection{$x-\frac{x^2}{2}<\ln(1+x)<x;\forall x>0$}
\begin{gather*}
	f(x)=\ln(1+x) \\
	f'(x)=\frac{1}{1+x} \\
	\forall x,\exists c\in(0,x):f'(c)=\frac{f(x)-f(0)}{x-0} \\
	\frac{1}{1+c}=\frac{\ln(1+x)-\ln(1)}{x} \\
	\frac{x}{1+c}=\ln(1+x) \\
	\forall c\in(0,x),\frac{1}{1+c}<1\Rightarrow \ln(1+x)<x \\
	\line(1, 0){125} \\
	f(x)=\ln(1+x)-x+\frac{x^2}{2} \displaybreak \\
	f'(x)=\frac{1}{1+x}+x-1  \\
	\exists c\in(0,x):f'(c)=\frac{f(x)-f(0)}{x-0} \\
	\frac{1}{1+c}+c-1=\frac{\ln(1+x)-x+\frac{x^2}{2}}{x} \\
	x(\frac{1}{1+c}+c)-x=\ln(1+x)-x+\frac{x^2}{2} \\
	x(\frac{1}{1+c}+c)=\ln(1+x)+\frac{x^2}{2} \\
	\frac{1}{1+c}+c>1\Rightarrow \ln(1+x)+\frac{x^2}{2}>x \\
	\ln(1+x)>x-\frac{x^2}{2} \\
	\line(1, 0){125} \\
	\therefore \\
	\boxed{x-\frac{x^2}{2}<\ln(1+x)<x;\forall x>0}
\end{gather*}

\section{[\textcolor{red}{Leibniz's General Product Rule for Derivatives}] Let $f,g$ have $n^{th}$ order derivatives on $(a, b)$, where $f^{(k)}(c) ,g^{(k)}(c)$ denotes the $k^{th}$ order derivative of $f$ and $g$ at $c$, respectively. Also let $h=f\cdot g$. Show that for any $c\in(a,b)$,}
\begin{displaymath}
	h^{(n)}(c)=\sum_{k=0}^{n}\binom{n}{k}f^{(k)}(c)\cdot g^{(n-k)}(c)
\end{displaymath}
\begin{gather*}
	\text{Base case: }n=1 \\
	h^{(1)}(c)=f'(c)g(c)+f(c)g'(c) \\
	\sum_{k=0}^{1}\binom{1}{k}f^{(k)}(c)\cdot g^{(1-k)}(c)=f'(c)g(c)+f(c)g'(c) \\
	h^{(1)}(c)=\sum_{k=0}^{1}\binom{1}{k}f^{(k)}(c)\cdot g^{(1-k)}(c) \\
	\line(1, 0){125} \\
	\text{Inductive case: } \\
	\text{Let } d(F,G)=(FG)'=\sum_{k=0}^{1}\binom{1}{k}f^{(k)}(c)\cdot g^{(1-k)}(c) \displaybreak \\
	\begin{aligned}
		h^{(n+1)}(c)&=\paren[\bigg]{\sum_{k=0}^{n}\binom{n}{k}f^{(k)}(c)\cdot g^{(n-k)}(c)}' \\
		&=\sum_{k=0}^{n}\binom{n}{k}\paren[\bigg]{f^{(k)}(c)\cdot g^{(n-k)}(c)}' \\
		&=\sum_{k=0}^{n}\binom{n}{k}(f^{(k)}(c)\cdot g^{(n-k)}(c))' \\
		&=\sum_{k=0}^{n}\binom{n}{k}\paren[\bigg]{\sum_{k'=0}^{1}\binom{1}{k'}f^{(k+k')}(c)\cdot g^{(n-k+1-k')}(c)} \\
		&=\sum_{k=0}^{n}\binom{n}{k}f^{(k)}(c)\cdot g^{(n-k+1)}(c)+\binom{n}{k}f^{(k+1)}(c)\cdot g^{(n-k)}(c) \\
		&=\sum_{k=0}^{n+1}\binom{n+1}{k}f^{(k)}(c)\cdot g^{((n+1)-k)}(c) \\
	\end{aligned} \\
	\line(1, 0){125} \\
	\text{By Induction} \\
	\therefore \\
	\boxed{h^{(n)}(c)=\sum_{k=0}^{n}\binom{n}{k}f^{(k)}(c)\cdot g^{(n-k)}(c)}
\end{gather*}

\end{document}